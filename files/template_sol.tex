\documentclass{article}
\usepackage[bulgarian]{babel}
\usepackage{amsmath}
\usepackage{geometry}

\geometry{a4paper, margin=1in}


\begin{document}
% Чувствайте се свободни да правите леки изменения във форматирането, но все пак гледайте да прилича на другите задачи

\section*{<Име на задачата>}
% Тук напишете условието на задачата
% Ако условието ви включва големи изображения или таблици може да ги сложите на отделен лист, след условието и решението

\begin{flushright}
\textit{\textbf{<Име на автора + клас (незадължително)>} е в настроение за\dots}

<Тема>, \today
\end{flushright}

\textbf{Отговор: <Числен и/или буквен>} % Отговор

\section*{Решение:}
% Тук напишете решението на задачата
% Ако решението ви включва големи изображения или таблици може да ги сложите на отделен лист, след условието и решението

\end{document}
